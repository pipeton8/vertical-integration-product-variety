\documentclass[10pt,aspectratio=169,english]{beamer}

\usepackage{multirow}
\usepackage{mathpazo}
\usepackage{natbib}
\usepackage{setspace}
\usepackage{tikz}
\usepackage{babel}
\usepackage[utf8]{inputenc}
\usepackage[T1]{fontenc}
\usepackage{subcaption}
\usepackage{ragged2e}
\usepackage{color, colortbl}
\usepackage{xcolor}
\usepackage{mathptmx}
\usepackage{multirow}
\usepackage{bbm}
\usepackage[scale=1.1]{FiraSans}

% Theme, color and font
% \usetheme{focus}
\usetheme{Singapore}
\usecolortheme{seagull}
% \usefonttheme{professionalfonts}
\beamertemplatenavigationsymbolsempty
\setbeamertemplate{frametitle}[default][left]

% Bibliography
\urlstyle{same}
% \addbibresource{references.bib}
\bibliographystyle{apa-good}

% Custom bibliography item
\setbeamertemplate{bibliography item}[triangle]

\setbeamerfont{title}{size=\huge, shape=\scshape\bfseries}
\setbeamerfont{subtitle}{size=\Large, shape=\scshape, parent=structure}
\setbeamerfont{author}{size=\Large, shape=\scshape}

\setbeamerfont{institute}{size=\large, shape=\scshape}
\setbeamerfont{date}{size=\large, shape=\scshape}

\setbeamerfont{sectiontitle}{size=\huge, series=\scshape\bfseries}
\setbeamerfont{frametitle}{size=\Large, family=\bfseries, shape=\scshape}

\setbeamerfont{footline}{size=\scriptsize}

\setbeamerfont{focusframe}{size=\huge, shape=\scshape}

\setbeamerfont{description item}{shape=\bfseries}

\setbeamerfont{caption name}{shape=\bfseries}

\setbeamerfont{bibliography item}{size=\small, shape=\scshape}
\setbeamerfont{bibliography entry author}{size=\small, shape=\scshape}
\setbeamerfont{bibliography entry title}{size=\small, series=\scshape\bfseries}
\setbeamerfont{bibliography entry location}{size=\small, shape=\scshape\normalfont}
\setbeamerfont{bibliography entry note}{size=\small, shape=\scshape\normalfont}

% Colors
\definecolor{urlcolor}{rgb}{0,.145,.698}
\definecolor{linkcolor}{rgb}{0,0,0}
\definecolor{citecolor}{rgb}{.12,.54,.11}
\definecolor{emphRed}{RGB}{229, 80, 57}

% Graphics path
% \graphicspath{../../figures/}

% Custom commands
\newcommand{\emphRed}[1]{\textcolor{emphRed}{#1}}

% Hyperref setup
\hypersetup{
	breaklinks=true,  % so long urls are correctly broken across lines
		colorlinks=true,
		urlcolor=urlcolor,
		linkcolor=linkcolor,
		citecolor=citecolor,
}

% Colors for table
\newcommand{\blue}[1]{\textcolor{blue}{#1}}

%:Title elements
\title{Are Acquisitions Intended for Specialization or Diversification? \\ {\LARGE Evidence from the Videogame Industry}}
\subtitle{VII Alumni Workshop}
\author[Felipe Del Canto]{Felipe Del Canto}
\institute[]{Harvard University}
\date{December 22, 2025}

%:Document
\begin{document}

%:Title
\frame[plain]{\titlepage}

% \section{Motivation}

% \begin{frame}{Consumer Welfare and Product Variety}
% 	\begin{itemize}
% 		\item Relevant questions focus around understanding the welfare effects of\vspace{0.5ex}
% 			\begin{itemize}
% 				\item Mergers and acquisitions.\vspace{0.5ex}
% 				\item Entry and exit.\vspace{0.5ex}
% 				\item Policy changes.
% 			\end{itemize}
% 		\vfill
		
% 		\item This entails understanding the (equilibrium) responses of:\vspace{0.5ex}
% 			\begin{enumerate}
% 				\item Prices.\vspace{0.5ex}
% 				\item Number of products.\vspace{0.5ex}
% 				\item Product variety (e.g., quality).
% 			\end{enumerate}
% 		\vfill

% 		\item I will focus on the effects of vertical integration (VI) on product variety.
% 	\end{itemize}
% \end{frame}

% \begin{frame}{How can VI impact Product Variety?} \hypertarget{motivation:VI-and-product-composition}{}
% 	\begin{itemize}
% 		\item Product variety can have two dimensions:\vspace{0.5ex}
% 			\begin{enumerate}
% 				\item Vertical (e.g., quality).\vspace{0.5ex}
% 				\item Horizontal (e.g., product characteristics).
% 			\end{enumerate}
% 		\vfill

% 		\item VI may affect the vertical dimension through investment.\vspace{0.5ex}
% 			\begin{itemize}
% 				\item Integration can lead to more investment (e.g. alleviating the hold-up problem).\vspace{0.5ex}
% 				\item But can also reduce incentives to invest \citep[see][]{Whinston2003}.\vspace{0.5ex}
% 				\item On top, VI can change investment decisions of competitors in equilibrium.
% 			\end{itemize}
% 		\vfill

% 		\item VI can affect horizontal dimensions through acquired firm choices.\vspace{0.5ex}
% 			\begin{itemize}
% 				\item Integration can happen in different locations in characteristic space.\vspace{0.5ex}
% 				\item Diversification/specialization $\Rightarrow$ different product offerings in equilibrium.\vspace{1ex}
				
% 				\hspace{-2ex}\hyperlink{intuition-characteristic-space}{\beamerbutton{Graphical intuition}}
% 			\end{itemize}
% 	\end{itemize}
% \end{frame}

\begin{frame}{This project}
	\begin{enumerate}
		\item \textbf{Today:} Analyze whether firms diversify/specialize when integrating.\vspace{0.5ex}
			\begin{itemize}
				\item In the context of the videogame industry: where publishers acquire developers.\vspace{0.5ex}
				\item Specifically, the market for (digital) PC games on Steam.\vspace{0.5ex}
				\item For now, focus on 25 major acquisitions between 2007 and 2024.\vspace{0.5ex}
				\item To measure specialization, focus on the genre of games.
			\end{itemize}
		\vfill

		\item \textbf{Next steps:} Study how integration impacts equilibrium product variety.\vspace{0.5ex}
			\begin{itemize}
				\item Build a structural model of vertical integration.\vspace{0.5ex}
				\item In the model, upstream firms are highly specialized single-product firms.\vspace{0.5ex}
				\item The distribution of characteristics depends on the identity of acquired firms.
			\end{itemize}
		% \vfill

		% \item Study the characteristics distribution across different counterfactuals.\vspace{0.5ex}
		% 	\begin{itemize}
		% 		\item Banning vertical integration.\vspace{0.5ex}
		% 		\item Socially optimal integration strategies.
		% 	\end{itemize}
	\end{enumerate}
\end{frame}

% \begin{frame}{Literature review}
% 	\begin{itemize}
% 		\item Endogenous product choice.\vspace{1ex}
% 			\begin{itemize}
% 				\item \citet{Fan2013,MazzeoEtAl2018,Wollmann2018,FanYang2020,Sullivan2020,BerryWaldfogel2001}.
% 			\end{itemize}
% 		\vfill

% 		\item Vertically integrated markets/Exclusive agreements.\vspace{1ex}
% 			\begin{itemize}
% 				\item \citet{Chen2007,ConlonMortimer2013,Lee2013,Asker2016,CrawfordEtAl2018}.
% 			\end{itemize}
% 		\vfill

% 		\item Videogame industry.\vspace{1ex}
% 			\begin{itemize}
% 				\item \citet{Nair2007,Lee2013,GilWarzynski2015,RusakovKretschmer2024,ArgyresEtAl2025}.
% 			\end{itemize}
% 	\end{itemize}
% \end{frame}

% \section{Industry Details}

% \begin{frame}{The videogame industry}
% 	\begin{itemize}
% 		\item The market for videogames is sizable.\vspace{0.5ex}
% 			\begin{itemize}
% 				\item Global revenues of $\sim$183 billion USD in 2024, $>50\%$ from mobile (Newzoo, 2024).\vspace{0.5ex}
% 				\item Digital games have become the norm (especially on PC).\vspace{1ex}
% 			\end{itemize}
% 		\vfill

% 		\item Three main agents:\vspace{0.5ex}
% 			\begin{itemize}
% 				\item \textbf{Developers}: Create videogames.\vspace{0.5ex}
% 				\item \textbf{Publishers}: Finance, provide game testing services, and distribute the game.\vspace{0.5ex}
% 				\item \textbf{Stores}: Sell games to consumers (e.g., Steam).
% 			\end{itemize}
% 		\vfill

% 		\item Vertical integration happens when publishers acquire developers.\vspace{0.5ex}
% 			\begin{itemize}
% 				\item The industry has seen a large number of acquisitions.\vspace{0.5ex}
% 				\item Mostly due to increases in game development costs.
% 			\end{itemize}
% 	\end{itemize}
% \end{frame}

% \begin{frame}{Why This is a Good Setting}
% 	\begin{itemize}
% 		\item Studios are highly specialized single-product firms.\vspace{0.5ex}
% 			\begin{itemize}
% 				\item Studios, in general, develop one game at a time and within a specific genre.\vspace{0.5ex}
% 				\item Development cycles last multiple years.
% 			\end{itemize}
% 		\vfill

% 		\item Developers contract with only one publisher, even without integration.\vspace{0.5ex}
% 			\begin{itemize}
% 				\item Foreclosure incentives in this case might only work through talent retention.
% 			\end{itemize}
% 		\vfill

% 		\item Contracts between developers and publishers are fairly homogeneous.\vspace{0.5ex}
% 			\begin{itemize}
% 				\item Publishers provide the same services in general.\vspace{0.5ex}
% 				\item Funding: milestone payments and revenue sharing \citep{GilWarzynski2015}. 
% 			\end{itemize}
% 	\end{itemize}
% \end{frame}

% \begin{frame}{Steam and Valve}
% 	\begin{itemize}
% 		\item Steam is the largest digital distribution platform for PC games.\vspace{0.5ex}
% 			\begin{itemize}
% 				\item Launched in 2003 by Valve Corporation.\vspace{0.5ex}
% 				\item Over 180 million (average) monthly active users in 2024 (Epyllion, 2025).\vspace{0.5ex}
% 				\item Estimates suggest its market share in 2025 is more than 70\% (in revenue terms).
% 			\end{itemize}
% 		\vfill

% 		\item Steam has a fixed royalty rate of 30\% on game sales.\vspace{0.5ex}
% 			\begin{itemize}
% 				\item However, there is no fee for launching games on Steam.
% 			\end{itemize}
% 		\vfill

% 		\item Valve is also a developer/publisher.\vspace{0.5ex}
% 			\begin{itemize}
% 				\item Owns popular game franchises such as Half-Life, Counter-Strike, and Dota.\vspace{0.5ex}
% 				\item However, revenues from Steam come mostly from third-parties.\vspace{0.5ex}
% 				\item Thus, scope for strategic behavior seems limited.
% 			\end{itemize}
% 	\end{itemize}
% \end{frame}

% \section{Data}

% \begin{frame}{\textbf{Data: Overview}}
% 	\begin{itemize}
% 		\item Dataset features the universe of videogames in the Steam store.\vspace{0.5ex}
% 			\begin{itemize}
% 				\item A total of $\sim$106,000 games.\vspace{0.5ex}
% 				\item Will only consider those launched between 2007 and 2025 (104,575 games).
% 			\end{itemize}
% 		\vfill
		
% 		\item For each game: developer(s) and publisher(s), release date and genre.\vspace{0.5ex}
% 			\begin{itemize}
% 				\item Among others (e.g., prices, estimated sales, reviews).
% 			\end{itemize}
% 		\vfill

% 		\item For today, use genre as horizontal characteristic.\vspace{0.5ex}
% 			\begin{itemize}
% 				\item Games are associated with one or more genres.\vspace{0.5ex}
% 				\item The dataset features 33 different genres.
% 			\end{itemize}
% 	\end{itemize}
% \end{frame}

\begin{frame}{\textbf{Large Acquisitions of Specialized Developers}} \hypertarget{data:acquisitions}{} 
	\vspace{1.5ex}
	\begin{figure}[H]
		\centering
		\begin{subfigure}{0.495\textwidth}
			\includegraphics[width=\textwidth]{../../figures/motivating facts/acquisitions-timeline-avgValue.pdf}
			\caption{Acquisition value (billions of 2024 USD)}
		\end{subfigure} \hfill
		\begin{subfigure}{0.495\textwidth}
			\includegraphics[width=\textwidth]{../../figures/genre distribution/developer-publisher-genre-distribution-histogram-inSample.pdf}
			\caption{Genre distribution.}
		\end{subfigure}
	\end{figure}
\end{frame}

% \begin{frame}{Developers are more Specialized than Publishers} \hypertarget{data:developer-publisher-specialization}{}
% 	\begin{figure}[H]
% 		\centering
% 		\begin{subfigure}{0.495\textwidth}
% 			\includegraphics[width=\textwidth]{../../figures/genre distribution/developer-publisher-genre-distribution-histogram.pdf}
% 			\caption{Full sample (with at least 4 games).}
% 		\end{subfigure} \hfill
% 		\begin{subfigure}{0.495\textwidth}
% 		\end{subfigure}
% 	\end{figure}

	% \quad \hyperlink{appendix:developer-summary-statistics}{\beamerbutton{Developers summary statistics}} \hyperlink{appendix:publisher-summary-statistics}{\beamerbutton{Publishers summary statistics}}
% \end{frame}

% \section{Results}

\begin{frame}{A measure of specialization/diversification}
	\begin{columns}
		\begin{column}{0.35\textwidth}
			Define the ``genre portfolio'' of $i$ as the vector $p_{i}$ with 
				$$ p_{ig} = \frac{\text{games of $i$ in genre $g$}}{\text{games of $i$}}$$
			\vspace{2ex}
				
			For publisher $p$ and developer $d$ define their similarity as
				$$ s_{pd} = \frac{p_{p} \cdot p_{d}}{||p_{p}|| \cdot ||p_{d}||} $$
			
			$\uparrow s_{pd}$ $\Rightarrow$ more specialization.
		\end{column}
		\onslide<2->{
		\begin{column}{0.65\textwidth}
			\begin{figure}[H]
				\centering
				\vspace{1.5ex}
				\includegraphics[width=\textwidth]{../../figures/specializations/genre_similarity_timeline.pdf}
			\end{figure}
		\end{column}
		}
	\end{columns}
\end{frame}

% \begin{frame}{Overall, more specialization than diversification}
% 	\begin{figure}[H]
% 		\centering
% 		\vspace{1.5ex}
% 		\includegraphics[width=0.7\textwidth]{../../figures/specializations/genre_similarity_timeline.pdf}
% 	\end{figure}
% \end{frame}

% \begin{frame}{Determinants of specialization}
% 	\begin{itemize}
% 		\item Which features of the market predict specialization?
% 		\vfill

% 		\item In terms to the publisher's portfolio, I test two competing hypotheses:\vspace{0.5ex}
% 			\begin{enumerate}
% 				\item (``Learning-by-doing'') More games in a genre $\Rightarrow$ more specialization.\vspace{0.5ex}
% 				\item (Cannibalization) More games in a genre $\Rightarrow$ more diversification.
% 			\end{enumerate}
% 		\vfill

% 		\item For market conditions, I test if more crowded genres predict diversification.\vspace{0.5ex}
% 			\begin{itemize}
% 				\item More competitive genre $\Rightarrow$ should avoid acquiring a developer specialized in it.
% 			\end{itemize}
% 	\end{itemize}
% \end{frame}

\begin{frame}{Determinants of Specializations}
	\begin{columns}
		\begin{column}{0.4\textwidth}
			Which features of the market predict whether publishers specialize when integrating?\vspace{5ex}

			Regarding the publisher's portfolio, test two hypotheses:\hspace{-1ex}\vspace{0.5ex}
			\begin{enumerate}
				\item Having more games in a genre $\Rightarrow$ specialization.\vspace{0.5ex}
				\item Competitors have more games in a genre $\Rightarrow$  diversification.
			\end{enumerate} 
		\end{column}
		\onslide<2->{
		\begin{column}{0.6\textwidth}
			\begin{table}
        \centering
		\resizebox{\textwidth}{!}{
        \begin{tabular}{lcc}
                \hline\hline
                \\[-1ex]
                        &       \multicolumn{2}{c}{Specialization ($\mathbbm{1}\{s_{pd} > 0.5$\})}              \\[0.5ex] \cline{2-3}
                \\[-1ex]
            &   (1)     & (2)  \\[0.5ex] \hline
             \\[-1ex]
\# of games by publisher & 0.010$^{***}$ & 0.012$^{***}$  \\[0.25ex]
 in top genre of developer & \footnotesize{(0.002)} & \footnotesize{(0.003)}        \\[1ex]
\# of games in market & -0.086$^{***}$ & -0.120$^{***}$      \\[0.25ex]
 in top genre of developer & \footnotesize{(0.023)} & \footnotesize{(0.027)}        \\[1ex]
                 \hline
                 \\[-1ex]
                Acquisition year FE &  & $\checkmark$ \\[1ex]
Observations  & 21 & 16 \\[1ex]
                \\[-2ex]
                \hline \hline
                \\[-1ex]
                 \multicolumn{3}{l}{$^{***}$ $p<0.01$, $^{**}$ $p<0.05$, $^{*}$ $p<0.1$.}      
        \end{tabular}
		}
		\end{table}
		\end{column}
		}
	\end{columns}
\end{frame}

% \begin{frame}{The overall market looks more specialized}
% 	\begin{figure}[H]
% 		\centering
% 		\includegraphics[width=0.7\textwidth]{../../figures/developer distribution/developer-tsne-distribution.pdf}
% 	\end{figure}
% \end{frame}

\begin{frame}{Conclusion}
	\begin{itemize}
		\item How does vertical integration impact product variety?\vspace{0.5ex}
			\begin{itemize}
				\item Specialization/diversification might lead to different equilibrium outcomes.\vspace{0.75ex}
				\item Understanding this phenomenon is relevant for competition/consumer welfare.
			\end{itemize}
		\vfill

		\item In the videogame industry, publishers tend to specialize with acquisitions.
		\vfill

		\item Specialization is more likely when:\vspace{0.5ex}
			\begin{enumerate}
				\item Publishers have more experience in the developer's genre (``learning-by-doing'').\vspace{0.75ex}
				\item The developer's genre is less crowded (less competition).
			\end{enumerate}  
	\end{itemize}
\end{frame}

\appendix

\begin{frame}[plain,c]
	\begin{center}
		\vspace{5ex}
		\Huge\bf THANK YOU!
		% Print slide number
		% \insertframenumber
	\end{center}
\end{frame}

\begin{frame}[allowframebreaks, plain]
	\frametitle{\bf References}
	\def\newblock{\hskip .11em plus .33em minus .07em}
	\nocite{*}
	\bibliography{../../references.bib}
\end{frame}

\appendix

\section{Appendix: Additional Figures and Tables}

\begin{frame}{Firm Distribution has Different implications \hyperlink{motivation:VI-and-product-composition}{\beamerbutton{Back}}} \hypertarget{intuition-characteristic-space}{} 
	\begin{figure}
		\begin{subfigure}{0.495\textwidth}
			\includegraphics[width=\textwidth]{../../figures/specialization intuition/distribution-diversified.pdf}
			\caption{Equilibrium with diversification}
		\end{subfigure} \hfill
		\begin{subfigure}{0.495\textwidth}
			\includegraphics[width=\textwidth]{../../figures/specialization intuition/distribution-specialized.pdf}
			\caption{Equilibrium with specialization}
		\end{subfigure} 
	\end{figure}
\end{frame}

\begin{frame}{\textbf{Developers: Summary statistics} \hyperlink{data:developer-publisher-specialization}{\beamerbutton{Back to data}}} \hypertarget{appendix:developer-summary-statistics}{}
	\begin{table}
     \small
     \centering
     \begin{tabular}{lcccc}
             \hline\hline
             \\[-1.5ex]
                     &       Mean    &       SD &    Min     &       Max     \\[0.5ex] \hline
             \\[-1.5ex]
             \multicolumn{5}{l}{\textbf{Panel (a): Full sample (N = 63,811)}}                           \\[0.5ex]
\quad Number of Games   &       1.66      &       3.17      &       1.00      &       203.00      \\[0.5ex]
\quad Estimated Sales (millions)   &       0.18      &       2.87      &       0.01      &       510.00      \\[0.5ex]
\quad Metacritic Score (\%)   &       72.69      &       10.20      &       6.00      &       97.00      \\[0.5ex]
\quad Positive Reviews (thousands)   &       2.50      &       64.89      &       0.00      &       12633.50      \\[0.5ex]
\quad Negative Reviews (thousands)   &       0.50      &       11.91      &       0.00      &       1824.82      \\[0.5ex]
\\[-1.5ex]
             \multicolumn{5}{l}{\textbf{Panel (b): Acquired (N = 25)}}                              \\[0.5ex]
\quad Number of Games   &       9.00      &       6.80      &       1.00      &       25.00      \\[0.5ex]
\quad Estimated Sales (millions)   &       13.38      &       32.09      &       0.02      &       162.65      \\[0.5ex]
\quad Metacritic Score (\%)   &       78.09      &       5.64      &       65.00      &       88.00      \\[0.5ex]
\quad Positive Reviews (thousands)   &       131.74      &       216.50      &       0.01      &       1100.76      \\[0.5ex]
\quad Negative Reviews (thousands)   &       29.28      &       75.27      &       0.00      &       378.54      \\[0.5ex]
             \\[-2ex]
             \hline \hline
     \end{tabular}
\end{table}

\end{frame}

\begin{frame}{\textbf{Publishers: Summary statistics} \hyperlink{data:developer-publisher-specialization}{\beamerbutton{Back to data}}} \hypertarget{appendix:publisher-summary-statistics}{}
	\begin{table}
     \small
     \centering
     \begin{tabular}{lcccc}
             \hline\hline
             \\[-1.5ex]
                     &       Mean    &       SD &    Min     &       Max     \\[0.5ex] \hline
             \\[-1.5ex]
             \multicolumn{5}{l}{\textbf{Panel (a): Full sample (N=56,688)}}                             \\[0.5ex]
\quad Number of Games   &       1.86      &       5.57      &       1.00      &       538.00      \\[0.5ex]
\quad Estimated Sales (millions)   &       0.20      &       3.79      &       0.01      &       510.35      \\[0.5ex]
\quad Metacritic Score (\%)   &       72.61      &       10.13      &       6.00      &       97.00      \\[0.5ex]
\quad Positive Reviews (thousands)   &       2.84      &       76.55      &       0.00      &       12634.79      \\[0.5ex]
\quad Negative Reviews (thousands)   &       0.58      &       15.10      &       0.00      &       1825.02      \\[0.5ex]
\\[-1.5ex]
             \multicolumn{5}{l}{\textbf{Panel (b): Acquirers (N=15)}}                               \\[0.5ex]
\quad Number of Games   &       69.87      &       55.67      &       1.00      &       142.00      \\[0.5ex]
\quad Estimated Sales (millions)   &       66.53      &       90.33      &       0.01      &       349.27      \\[0.5ex]
\quad Metacritic Score (\%)   &       73.22      &       7.62      &       51.00      &       80.00      \\[0.5ex]
\quad Positive Reviews (thousands)   &       782.77      &       1006.13      &       0.02      &       3670.00      \\[0.5ex]
\quad Negative Reviews (thousands)   &       162.49      &       282.17      &       0.01      &       1111.46      \\[0.5ex]
             \\[-2ex]
             \hline \hline
     \end{tabular}
\end{table}

\end{frame}

\end{document}

