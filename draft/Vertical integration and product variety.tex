\documentclass[english]{article}

\usepackage[utf8]{inputenc}
\usepackage[T1]{fontenc}
\usepackage[margin=1in]{geometry}
\usepackage{threeparttable}
\usepackage{float}
\usepackage{color}
\usepackage[usenames,dvipsnames,svgnames]{xcolor} 
\usepackage{mathtools}
\usepackage{multirow}
\usepackage{hyperref}
\usepackage{cleveref}
\usepackage{natbib}
\usepackage{amsfonts}
\usepackage{caption}
\usepackage{subcaption}
\usepackage{amsmath, amssymb, amsthm}
\usepackage{soul}
\usepackage{tabularray}
\usepackage{bbm}
\usepackage{tikz}
\usepackage{setspace}
\usepackage{enumitem}

% Spacing
\setstretch{1.2}

% Environments
\newtheorem{proposition}{Proposition}

% Bibliography style
\bibliographystyle{apa-good} 
\urlstyle{same}

% Colors
\definecolor{urlcolor}{rgb}{0,.145,.698}
\definecolor{linkcolor}{rgb}{.71,0.21,0.01}
\definecolor{citecolor}{rgb}{.12,.54,.11}
		
\hypersetup{
	breaklinks=true,  % so long urls are correctly broken across lines
	colorlinks=true,
	urlcolor=urlcolor,
	linkcolor=linkcolor,
	citecolor=citecolor,
}

% Packages to add code
\usepackage{beramono}
\usepackage{listings}
\usepackage{upquote}

% Parameters
\title{\textbf{Product composition in vertically integrated markets}}
\author{Felipe del Canto}
\date{\today}

\begin{document}

\maketitle

\begin{abstract}
	The study of the welfare effects of vertical integration has focused on two main forces. On the one hand, integration can increase efficiency by reducing double marginalization and aligning investment incentives. On the other, it can lead to foreclosure and increased incentives to raise rivals' costs. However, the endogenous decision of which firms to acquire can also have important implications for product composition in equilibrium, both in horizontal and vertical dimensions. I investigate this issue in the market for videogames, where publishers (downstream) can acquire developers (upstream). Using a detailed dataset with the near universe of developers and publishers since the 90s, I first document whether publishers' acquisitions specialize or diversify their game portfolios in characteristic space. Then, using data on game releases I analyze whether integration leads the acquired developers to increase their quality, using reviews as a proxy. Fianlly, I attempt to rationalize the findings using a static model where a monopolistic publisher can acquire a developer from a pool of horizontally differentiated firms.
\end{abstract}

\section{Introduction}

\section{Model}
\label{sec:model}

This section lays out a static, closed-form model of product composition in a vertically related industry without an explicit ``store'' layer. The primitives are kept intentionally simple in the spirit of early IO: linear demands with quality in the intercept, constant marginal costs, and a quadratic investment technology. Governance (self-publish, contract, or integrate) affects the efficiency of quality investment and which titles' prices are jointly chosen, thereby determining how much within-portfolio cannibalization is internalized.

\subsection{Agents, products, and buckets}

There are developers $d \in \mathcal D$ (each with one project), and publishers $P \in \mathcal P$ (each may already control a portfolio $\mathcal J_P$ of titles). A ``title'' $i$ belongs to a bucket (genre/tag) $g(i)\in\mathcal G$. 

Each title $i$ has marginal cost $c_i > 0$ and a scalar latent vertical attribute (``quality'') $a_i \ge 0$. Let $\bar a_i \ge 0$ denote the baseline quality before investment. When a title is governed under regime $s\in\{\mathbf S,\mathbf C,\mathbf I\}$ (Self-publish, Contract, Integrate), quality is produced by investment $x_i \ge 0$ according to
\begin{equation}
a_i \;=\; \bar a_i + \theta_s\, x_i, 
\qquad \text{with cost } \; C_s(x_i)=\frac{\kappa_s}{2}\,x_i^2,
\label{eq:quality_tech}
\end{equation}
where $(\theta_s,\kappa_s)$ capture investment efficiency under governance $s$. We allow $\theta_{\mathbf I}\ge \theta_{\mathbf C} \ge \theta_{\mathbf S}$ and/or $\kappa_{\mathbf I}\le \kappa_{\mathbf C}\le \kappa_{\mathbf S}$ to reflect financing/QA/coordination advantages under integration/contracting.

\paragraph{Who sets prices and invests.} Under $\mathbf S$ the developer invests and sets its own price. Under $\mathbf C$ the publisher invests and sets prices for the contracted title. Under $\mathbf I$ the publisher invests and sets prices for the acquired title. In all cases, the \emph{pricing entity} chooses prices to maximize the sum of operating profits across all titles it prices; this is the sole channel through which cannibalization is internalized.

\subsection{Demand with cross- and within-bucket substitution}

Time is static and market size is normalized so that linear inverse demands aggregate to the following direct demands. For titles $i$ and $j$ in buckets $g(i)$ and $g(j)$, quantities are
\begin{align}
q_i &= A_{g(i)} + \gamma a_i \;-\; b\,p_i + s_{g(i),g(j)}\, p_j, \label{eq:lin_demand_i}\\
q_j &= A_{g(j)} + \gamma a_j \;-\; b\,p_j + s_{g(j),g(i)}\, p_i, \label{eq:lin_demand_j}
\end{align}
with $b>0$, $\gamma>0$ and $s_{g,h}\ge 0$ the \emph{substitutability} parameters. We distinguish \emph{within-bucket} substitution $s_{w}\equiv s_{g,g}$ and \emph{cross-bucket} substitution $s_c\equiv s_{g,h}$ for $g\neq h$, typically $s_w \ge s_c$. The $A_g$ are bucket-specific demand shifters. For larger portfolios, the demands extend in the standard way with cross-price terms $s_{g(i),g(k)} p_k$ for all other titles $k$.

\subsection{Timing of the game}

The one-shot game for a focal developer $d$ with project $i$ proceeds as follows:
\begin{enumerate}[label=\Roman*.]
\item \textbf{Information.} Primitives $(\bar a_i,c_i)$ and the substitutability structure $\{A_g,b,\gamma,s_w,s_c\}$ are common knowledge. Each publisher $P$ has an existing priced portfolio $\mathcal J_P$ (possibly empty).
\item \textbf{Governance offers (menus).} Each publisher $P$ simultaneously offers a menu to $d$: a contract option $\mathbf C$ with revenue share $\alpha_P\in[0,1]$ and/or fixed transfer $F_{C,P}$, and an integration option $\mathbf I$ with acquisition transfer $F_{I,P}$. Transfers only split surplus and do not affect operating decisions. If no offer is accepted, the default is $\mathbf S$.
\item \textbf{Acceptance and bucket assignment.} The developer selects an offer/regime, or self-publishes. The pricing entity for $i$ (developer under $\mathbf S$, publisher under $\mathbf C$ or $\mathbf I$) then assigns the title to a bucket $g(i)\in\mathcal G$.
\item \textbf{Investment.} The investing party chooses $x_i\ge 0$ and induces $a_i$ via \eqref{eq:quality_tech}.
\item \textbf{Pricing.} Each pricing entity sets prices $\{p_k\}_{k\in \mathcal J_F}$ for the titles $\mathcal J_F$ it controls to maximize the sum of operating profits $\sum_{k\in\mathcal J_F} (p_k-c_k) q_k$ given demands \eqref{eq:lin_demand_i}--\eqref{eq:lin_demand_j}.
\item \textbf{Payoffs.} Operating profits minus investment costs and plus/minus transfers realize.
\end{enumerate}

\subsection{Pricing: closed forms in the two-title cases}

To keep results transparent, we state closed forms for the case of two titles $i$ and $j$ (possibly in different buckets) with $s\equiv s_{g(i),g(j)}=s_{g(j),g(i)}$. Define $\Delta \equiv -4b^2 + s^2 < 0$ under standard stability $s<2b$.

\paragraph{(i) Separate pricers (e.g., both self-publish).} Nash-Bertrand in prices yields
\begin{equation}
\begin{aligned}
p_i^{\text{Nash}} \;=\; \frac{-2b\big(A_{g(i)}+\gamma a_i\big) - s\,\big(A_{g(j)}+\gamma a_j\big) - 2b^2 c_i - b s\, c_j}{\Delta}, \\[0.5ex]
p_j^{\text{Nash}} \;=\; \frac{-2b\big(A_{g(j)}+\gamma a_j\big) - s\,\big(A_{g(i)}+\gamma a_i\big) - 2b^2 c_j - b s\, c_i}{\Delta}.	
\end{aligned}
\label{eq:price_nash}
\end{equation}

\paragraph{(ii) One multi-product pricer (contract or integrate).} If a single entity jointly prices $i$ and $j$, the multi-product FOCs give
\begin{equation}
\begin{aligned}
	p_i^{\text{MP}} \;=\; \frac{-b\big(A_{g(i)}+\gamma a_i\big) - s\,\big(A_{g(j)}+\gamma a_j\big) - b^2 c_i + s^2 c_i}{-2b^2 + 2 s^2}, \\[0.5ex]
	p_j^{\text{MP}} \;=\; \frac{-b\big(A_{g(j)}+\gamma a_j\big) - s\,\big(A_{g(i)}+\gamma a_i\big) - b^2 c_j + s^2 c_j}{-2b^2 + 2 s^2}.
\end{aligned}
\label{eq:price_mp}
\end{equation}
For substitutes ($s>0$) we have $p^{\text{MP}}\ge p^{\text{Nash}}$; the difference is larger when $s$ is larger, i.e., when titles sit in the same bucket ($s=s_w$) rather than across buckets ($s=s_c$). In both \eqref{eq:price_nash}--\eqref{eq:price_mp}, prices \emph{increase with quality} $a_i$ through the demand intercepts.

\subsection{Investment and quality}

Given prices (which depend on $a$), the operating profit for title $i$ is $\pi_i=(p_i-c_i)q_i$. Let $\varphi_i \equiv \partial \pi_i / \partial (A_{g(i)}+\gamma a_i)$ denote the marginal profit impact of the demand intercept. Under linear demand and either pricing regime, $\varphi_i$ is constant given $(p_j,a_j)$ and equals a simple function of $(b,s)$ and the regime. The investing party chooses $x_i$ to maximize
\begin{equation}
\pi_i\big(a_i(x_i)\big) - \frac{\kappa_s}{2}x_i^2
\quad\Rightarrow\quad
x_i^{*} \;=\; \frac{\theta_s \gamma}{\kappa_s}\,\varphi_i,
\qquad
a_i^{*} \;=\; \bar a_i \;+\; \theta_s x_i^{*}.
\label{eq:x_star}
\end{equation}
Thus $x_i^{*}$ is linear in the \emph{investment efficiency} $\theta_s/\kappa_s$ and in the pricing-return term $\varphi_i$, which (for substitutes) is \emph{larger} when prices are \emph{jointly internalized} by a multi-product pricer and when substitutability $s$ is higher.

\subsection{Regime choice (self-publish, contract, integrate)}

Let $U_d^{s}$ be the developer's operating profit net of its investment cost if governed by $s\in\{\mathbf S,\mathbf C,\mathbf I\}$, and $\Pi_P^{s}$ the publisher's corresponding operating profit net of investment for the publisher that would host the title under $s$ (for $\mathbf S$ set $\Pi_P^{\mathbf S}$ to the publisher's status quo profit). Define total surplus
\begin{equation}
TS_s \;\equiv\; U_d^{s} + \Pi_P^{s}.
\end{equation}
With simultaneous offers and Nash bargaining over transfers with developer weight $\beta\in[0,1]$, the realized match $(P^\star,s^\star)$ is \emph{efficient}:
\begin{equation}
(P^\star,s^\star)\in \arg\max_{P,\,s\in\{\mathbf C,\mathbf I\}} TS_s(P),
\end{equation}
provided $\max_{P,s} TS_s(P) \ge TS_{\mathbf S}$; otherwise the developer self-publishes. Transfers split the incremental surplus $TS_{s^\star}(P^\star)-TS_{\mathbf S}$ according to $\beta$.

Intuitively, $TS_s(P)$ is larger when (i) governance $s$ delivers more efficient investment ($\theta_s/\kappa_s$ higher), and (ii) the hosting publisher's portfolio $\mathcal J_P$ implies stronger cannibalization internalization at pricing (especially if the title would otherwise sit near existing titles within the same bucket).

\subsection{Results and testable predictions}

We collect the main conclusions as propositions for the two-title case; proofs are one-line substitutions using \eqref{eq:price_nash}--\eqref{eq:price_mp} and \eqref{eq:x_star}.

\begin{proposition}[Prices depend on quality and on governance]
\label{prop:price_quality}
Under either regime, $\partial p_i/\partial a_i>0$. Moreover, for substitutes ($s>0$), $p_i^{\text{MP}} \ge p_i^{\text{Nash}}$, with the gap increasing in $s$. Hence governance that consolidates pricing (contract or integration) softens competition the most when titles are in the \emph{same} bucket ($s=s_w$).
\end{proposition}

\begin{proposition}[Investment alignment under VI/contracting]
\label{prop:investment}
Optimal investment satisfies $x_i^{*}=(\theta_s\gamma/\kappa_s)\,\varphi_i$. If $\theta_{\mathbf I}/\kappa_{\mathbf I}>\theta_{\mathbf C}/\kappa_{\mathbf C}>\theta_{\mathbf S}/\kappa_{\mathbf S}$, then $x_i^{*}$ and $a_i^{*}$ are weakly highest under integration, next under contracting, and lowest under self-publishing. The investment difference is larger when $s$ is small (weaker substitution with close neighbors) and when the pricing entity does not internalize cannibalization.
\end{proposition}

\begin{proposition}[Bucket placement: specialization vs. diversification]
\label{prop:bucket}
Let $s_w$ and $s_c$ denote within- and cross-bucket substitution with $s_w>s_c\ge 0$. For a multi-title publisher that controls pricing, placing two in-house titles in the same bucket raises $\{p^{\text{MP}}\}$ but reduces $\{\varphi_i\}$ via stronger cannibalization, lowering $x_i^{*}$. If investment efficiency $(\theta_s/\kappa_s)$ is high, the publisher prefers to \emph{diversify} across buckets (use $s_c$) to preserve investment incentives; if it is low, the static pricing softening within a bucket (using $s_w$) can dominate. Hence stronger VI (higher $\theta_{\mathbf I}/\kappa_{\mathbf I}$) shifts portfolios toward diversification.
\end{proposition}

\begin{proposition}[Efficient regime selection]
\label{prop:selection}
With simultaneous menus and Nash bargaining, the realized governance is $s^\star=\arg\max \{TS_{\mathbf S},\max_{P}TS_{\mathbf C}(P),\max_{P}TS_{\mathbf I}(P)\}$. Therefore, integration/contracting occurs if and only if the incremental total surplus relative to self-publishing is positive; this is more likely when $(\theta_s/\kappa_s)$ is high and when the host publisher's portfolio implies material cannibalization internalization in pricing.
\end{proposition}

\paragraph{Empirical content.} The model yields (i) a single cross-bucket substitutability parameter $s_c$ and a within-bucket $s_w$ identified from price/quantity comovement, (ii) quality--price and quality--quantity positive slopes, (iii) stronger investment uplifts under VI/contract, and (iv) portfolio shifts toward less-crowded (cross-bucket) placements when governance consolidates pricing and raises investment efficiency.


\appendix
\section*{Appendix A: Proof sketches}
\label{app:proofs}
\paragraph{Proposition \ref{prop:price_quality}.} With two titles and $s\equiv s_{g(i),g(j)}$, the closed forms are given by \eqref{eq:price_nash}--\eqref{eq:price_mp}. Differentiating with respect to the own intercept $t_i\equiv A_{g(i)}+\gamma a_i$,
\[
\frac{\partial p_i^{\text{Nash}}}{\partial t_i}=\frac{-2b}{-4b^2+s^2}>0,\qquad
\frac{\partial p_i^{\text{MP}}}{\partial t_i}=\frac{-b}{-2b^2+2s^2}>0,
\]
under the stability restriction $s<2b$. Hence $\partial p_i/\partial a_i=\gamma\,\partial p_i/\partial t_i>0$. Comparing $p^{\text{MP}}$ vs. $p^{\text{Nash}}$ follows by evaluating the two linear systems (or by noting that joint pricing internalizes the cross-price term $s>0$ and therefore raises markups). The gap is increasing in $s$ because joint pricing captures a larger share of the diversion when substitution is stronger.

\paragraph{Proposition \ref{prop:investment}.} Let $\varphi_i\equiv \partial \pi_i/\partial t_i$ with $t_i=A_{g(i)}+\gamma a_i$. Using $q_i=t_i - b p_i + s p_j$ and $\pi_i=(p_i-c_i)q_i$,
\[
\varphi_i=(p_i-c_i)\Big(1 - b\,\frac{\partial p_i}{\partial t_i} + s\,\frac{\partial p_j}{\partial t_i}\Big)+ q_i\,\frac{\partial p_i}{\partial t_i}.
\]
Under both regimes, $\partial p_i/\partial t_i$ and $\partial p_j/\partial t_i$ are constants given above, so $\varphi_i$ is a constant given $(a_j,p_j)$. The investment problem is quadratic in $x_i$ with linear benefits $\theta_s\gamma\,\varphi_i$, giving \eqref{eq:x_star}. Because $\varphi_i$ is (for substitutes) larger under joint pricing and increases with $s$, investment is (weakly) higher when pricing is internalized and when $s$ is larger, holding $(\theta_s,\kappa_s)$ fixed; governance comparisons then follow from $\theta_s/\kappa_s$.

\paragraph{Proposition \ref{prop:bucket}.} Holding primitives fixed, moving two in-house titles from different buckets ($s_c$) into the same bucket ($s_w>s_c$) raises prices via \eqref{eq:price_mp} but reduces $\varphi_i$ only if the price effect dominates the intercept effect. With linear demand, the net effect on $\varphi_i$ is ambiguous in general; however, when investment efficiency $(\theta_s/\kappa_s)$ is high, the planner values preserving the marginal return to quality, which typically favors diversification (use $s_c$) in order to limit internal cannibalization in the denominator of $q_i$. The stated comparative static follows from substituting $s=s_w$ vs. $s_c$ into \eqref{eq:price_mp} and the expression for $\varphi_i$.

\paragraph{Proposition \ref{prop:selection}.} With simultaneous menus and Nash bargaining (developer weight $\beta$), realized transfers split the incremental surplus relative to self-publishing. Because transfers do not affect operating choices, equilibrium regime and match solve $\max_{P,s} TS_s(P)$ whenever this exceeds $TS_{\mathbf S}$; otherwise $\mathbf S$ obtains. This is the standard efficiency result for surplus-splitting games.

\section*{Appendix B: A small numerical illustration}
\label{app:numeric}
Parameters: $A_g=10$ for both buckets, $b=1$, $c_i=c_j=2$, $\gamma=1$, baseline quality $\bar a_i=\bar a_j=1$, within-bucket substitution $s_w=0.8$, cross-bucket substitution $s_c=0.3$. Investment technology: $(\theta,\kappa)$ equals $(0.5,1)$ under self-publishing $\mathbf S$, $(0.8,1)$ under contracting $\mathbf C$, and $(1,0.8)$ under integration $\mathbf I$. We consider two symmetric titles.
\begin{enumerate}[leftmargin=*,label=(\alph*)]
\item \textbf{Prices depend on quality and internalization.} With no investment ($a_i=a_j=\bar a=1$), the equilibrium prices and quantities are:
\begin{center}
\begin{tabular}{lcc}
\hline
Case & Price $p$ & Quantity $q$ \\
\hline
Same bucket ($s_w=0.8$), separate pricers (Nash) & $10.83$ & $8.83$ \\
Same bucket ($s_w=0.8$), joint pricer (MP)       & $28.50$ & $5.30$ \\
Split buckets ($s_c=0.3$), separate pricers (Nash) & $7.65$ & $5.65$ \\
Split buckets ($s_c=0.3$), joint pricer (MP)       & $8.86$ & $4.80$ \\
\hline
\end{tabular}
\end{center}
Thus $p^{\text{MP}}>p^{\text{Nash}}$ and $\partial p/\partial a>0$ in all cases.
\item \textbf{Investment levels.} Using $x_i^{*}=(\theta_s\gamma/\kappa_s)\,\varphi_i$ with the implied $\varphi_i$, the optimal investment (per title) is:
\begin{center}
\begin{tabular}{lccc}
\hline
Placement & $\mathbf S$ & $\mathbf C$ & $\mathbf I$ \\
\hline
Same bucket ($s_w$) & $5.26$ & $16.49$ & $25.76$ \\
Split buckets ($s_c$) & $2.89$ & $4.85$ & $7.58$ \\
\hline
\end{tabular}
\end{center}
Governance that both \emph{internalizes pricing} and improves investment efficiency (higher $\theta_s/\kappa_s$) yields the largest $x^*$; stronger substitution ($s_w>s_c$) further amplifies this.
\end{enumerate}
\medskip
These back-of-the-envelope values simply illustrate the comparative statics; the qualitative rankings are robust to wide parameter changes.

\nocite{*}
\bibliography{../references.bib}

\end{document}